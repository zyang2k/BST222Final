% Options for packages loaded elsewhere
\PassOptionsToPackage{unicode}{hyperref}
\PassOptionsToPackage{hyphens}{url}
% !TeX program = pdfLaTeX
\documentclass[12pt]{article}
\usepackage{amsmath}
\usepackage{graphicx,psfrag,epsf}
\usepackage{enumerate}
\usepackage[]{natbib}
\usepackage{textcomp}


%\pdfminorversion=4
% NOTE: To produce blinded version, replace "0" with "1" below.
\newcommand{\blind}{0}

% DON'T change margins - should be 1 inch all around.
\addtolength{\oddsidemargin}{-.5in}%
\addtolength{\evensidemargin}{-1in}%
\addtolength{\textwidth}{1in}%
\addtolength{\textheight}{1.7in}%
\addtolength{\topmargin}{-1in}%

%% load any required packages here



% tightlist command for lists without linebreak
\providecommand{\tightlist}{%
  \setlength{\itemsep}{0pt}\setlength{\parskip}{0pt}}



\usepackage{float}

\IfFileExists{bookmark.sty}{\usepackage{bookmark}}{\usepackage{hyperref}}
\IfFileExists{xurl.sty}{\usepackage{xurl}}{} % add URL line breaks if available
\hypersetup{
  pdftitle={Survival Analysis of Infection Control Measures in Burn Patients},
  hidelinks,
  pdfcreator={LaTeX via pandoc}}



\begin{document}


\def\spacingset#1{\renewcommand{\baselinestretch}%
{#1}\small\normalsize} \spacingset{1}


%%%%%%%%%%%%%%%%%%%%%%%%%%%%%%%%%%%%%%%%%%%%%%%%%%%%%%%%%%%%%%%%%%%%%%%%%%%%%%

\if0\blind
{
  \title{\bf Survival Analysis of Infection Control Measures in Burn
Patients}

  \author{
        Ziyue Yang \thanks{The authors gratefully acknowledge
Prof.~David Rocke and Brittany Lemmon for their instruction and guidance
throughout this project.} \\
    \\
      }
  \maketitle
} \fi

\if1\blind
{
  \bigskip
  \bigskip
  \bigskip
  \begin{center}
    {\LARGE\bf Survival Analysis of Infection Control Measures in Burn
Patients}
  \end{center}
  \medskip
} \fi

\bigskip
\begin{abstract}
TBD
\end{abstract}

\noindent%
 

\vfill

\newpage
\spacingset{1.9} % DON'T change the spacing!

\section{Introduction}\label{introduction}

Infection with \emph{Staphylococcus aureus} is a critical concern in
burn patients, often contributing to prolonged hospital stays, increased
morbidity, and higher healthcare costs \citep{norbury_infection_2016}.
Therefore, effective infection control measures are of great importance.
This study investigates the impact of replacing routine bathing with
total body washing using antimicrobial agents on infection risk,
leveraging survival analysis techniques to rigorously evaluate the time
to infection. The dataset, originally published by Ichida \emph{et al.}
(1993), provides data on infection times, patient characteristics, and
clinical interventions \citep{ichida_evaluation_1993}.

Survival analysis accommodates both censored observations (patients who
do not develop infections during the study) and the timing of events. In
particular, to highlight differences in infection-free survival between
patients receiving routine bathing and those undergoing total body
washing, the Kaplan-Meier estimator was used
\citep{kaplan_nonparametric_1958}.

To further identify factors influencing infection risk, we use the Cox
proportional hazards model \citep{cox1972}. This model evaluates the
relationship between various predictors---such as patient gender, burn
characteristics, and clinical treatments---and the likelihood of
infection over time. Both time-independent covariates (e.g.~patient
gender, race) and time-dependent covariates (e.g.~surgical excision of
burn tissue, prophylactic antibiotic treatment) are employed for a
comprehensive analysis of infection dynamics.

This report aims to present a clear and actionable evaluation of the
infection control measures through the lens of survival analysis, with
insights that inform clinical decision-making and contribute to better
patient care.

\section{Data Description}\label{data-description}

The dataset \texttt{burn} consists of 154 observations of burn patients
and 17 variables that capture patient characteristics, clinical
interventions, and infection status with time to infection.

\subsection{\texorpdfstring{\textbf{Outcome
Variables}}{Outcome Variables}}\label{outcome-variables}

\begin{itemize}
\tightlist
\item
  \textbf{T3 (time to infection)}: The time (in days) until infection
  with \emph{Staphylococcus aureus}.
\item
  \textbf{D3 (infection status)}: A binary variable indicating whether
  the patient developed an infection within the course of the study (1 =
  infected, 0 = not infected).
\end{itemize}

\subsection{\texorpdfstring{\textbf{Time-Dependent
Covariates}}{Time-Dependent Covariates}}\label{time-dependent-covariates}

\begin{itemize}
\tightlist
\item
  \textbf{T1 (time to surgical excision)}: The time (in days) to
  surgical excision of burn tissue.
\item
  \textbf{D1 (surgical excision status)}: A binary variable indicating
  whether surgical excision was performed (1 = excised, 0 = not
  excised).
\item
  \textbf{T2 (time to antibiotic treatment)}: The time (in days) to the
  administration of prophylactic antibiotic treatment.
\item
  \textbf{D2 (antibiotic treatment status)}: A binary variable
  indicating whether antibiotics were administered (1 = treated, 0 = not
  treated).
\end{itemize}

\subsection{\texorpdfstring{\textbf{Baseline
Characteristics}}{Baseline Characteristics}}\label{baseline-characteristics}

\begin{itemize}
\tightlist
\item
  \textbf{Treatment}: Categorical variable indicating the bathing
  regimen (Routine or Cleansing with antimicrobial agents).
\item
  \textbf{Gender}: Categorical variable indicating the patient's gender
  (Male or Female).
\item
  \textbf{Race}: Categorical variable indicating the patient's race
  (Nonwhite or White).
\item
  \textbf{PercentBurned}: Numeric variable representing the percentage
  of the patient's body surface area affected by burns.
\end{itemize}

\subsection{\texorpdfstring{\textbf{Burn Site
Characteristics}}{Burn Site Characteristics}}\label{burn-site-characteristics}

\begin{itemize}
\tightlist
\item
  \textbf{SiteHead}: Binary factor indicating whether the head was
  burned (Burned or Not Burned).
\item
  \textbf{SiteButtock}: Binary factor indicating whether the buttocks
  were burned (Burned or Not Burned).
\item
  \textbf{SiteTrunk}: Binary factor indicating whether the trunk was
  burned (Burned or Not Burned).
\item
  \textbf{SiteUpperLeg}: Binary factor indicating whether the upper leg
  was burned (Burned or Not Burned).
\item
  \textbf{SiteLowerLeg}: Binary factor indicating whether the lower leg
  was burned (Burned or Not Burned).
\item
  \textbf{SiteRespTract}: Binary factor indicating whether the
  respiratory tract was burned (Burned or Not Burned).
\end{itemize}

\subsection{\texorpdfstring{\textbf{Burn
Type}}{Burn Type}}\label{burn-type}

\begin{itemize}
\tightlist
\item
  \textbf{BurnType}: Categorical variable specifying the type of burn
  (Chemical, Scald, Flame, or Electric).
\end{itemize}

\section{Methods}\label{methods}

\subsection{\texorpdfstring{\textbf{Kaplan-Meier Survival
Analysis}}{Kaplan-Meier Survival Analysis}}\label{kaplan-meier-survival-analysis}

The Kaplan-Meier estimator was used to estimate and visualize the
probability of remaining infection-free over time for patients
undergoing either routine bathing or antimicrobial washing.
Additionally, Nelson-Aalen estimate was also plotted for a comprehensive
view of survival difference between groups.

Survival curves were compared using the \texttt{survdiff} function in
package \texttt{survival}, which performs a log-rank test to assess
whether the differences between the two groups are statistically
significant.

Additionally, cumulative hazard functions were plotted against time to
estimate the cumulative infection probability at different time points.
Complementary log-log survival curves were plotted against
log-transformed time to assess if the ratio of hazard rates between two
treatment groups remains constant over time.

\subsection{\texorpdfstring{\textbf{Cox Proportional Hazards
Model}}{Cox Proportional Hazards Model}}\label{cox-proportional-hazards-model}

\subsubsection{\texorpdfstring{\textbf{Model with Time-Independent
Covariates}}{Model with Time-Independent Covariates}}\label{model-with-time-independent-covariates}

An initial Cox proportional hazards model was constructed using
time-independent covariates to evaluate their relationship with the risk
of infection. The primary predictor of interest was \textbf{Treatment}.
Additional time-independent variables were sequentially introduced into
the model.

To address the violation of the proportional hazards assumption
identified in the unstratified model, the variable
\textbf{SiteRespTract} was stratified. This allowed for differing
baseline hazard functions for patients with and without burns in the
respiratory tract.

Model refinement was performed using the \texttt{drop1} function to
identify covariates that did not significantly contribute to model
performance. Multicollinearity among covariates was assessed using
variance inflation factors, ensuring that redundant predictors were
identified and addressed without compromising the integrity of the
model.

\subsubsection{\texorpdfstring{\textbf{Model with Time-Dependent
Covariates}}{Model with Time-Dependent Covariates}}\label{model-with-time-dependent-covariates}

To incorporate time-dependent predictors, the dataset was expanded using
counting process notation. Two key time-dependent covariates were
included: 1. \textbf{Surgical excision of burn tissue (T1, D1)}. 2.
\textbf{Prophylactic antibiotic treatment (T2, D2)}.

A Cox model was constructed combining time-dependent and
time-independent covariates. The time-dependent variables captured the
dynamic effects of these interventions on infection risk, while
time-independent variables provided baseline hazard adjustments.

\subsection{\texorpdfstring{\textbf{Model Checking and
Diagnostics}}{Model Checking and Diagnostics}}\label{model-checking-and-diagnostics}

Model checking was performed using a comprehensive suite of diagnostic
techniques:

\subsubsection{\texorpdfstring{\textbf{Proportional Hazards
Assumption}}{Proportional Hazards Assumption}}\label{proportional-hazards-assumption}

The proportional hazards assumption was evaluated using Schoenfeld
residual plots which test whether the residuals show systematic trends
over time and \texttt{cox.zph} function from package \texttt{survival}.
Where violations were detected, appropriate adjustments were made, such
as stratification to allow baseline hazard functions to differ between
groups.

\subsubsection{\texorpdfstring{\textbf{Goodness-of-Fit}}{Goodness-of-Fit}}\label{goodness-of-fit}

Cox-Snell residuals were used to assess overall model fit. A cumulative
hazard plot was constructed to compare observed data with the expected
hazard under the model. A straight 45-degree line indicated good fit,
while deviations suggested potential inadequacies.

\subsubsection{\texorpdfstring{\textbf{Outlier
Analysis}}{Outlier Analysis}}\label{outlier-analysis}

Martingale residuals and deviance residuals were plotted against the
linear predictor to assess nonlinearity and detect possible outliers in
the model.

\subsubsection{\texorpdfstring{\textbf{Influential
Observations}}{Influential Observations}}\label{influential-observations}

DFBETA values were plotted to identify leverage points, indicating
observations that had a disproportionately large influence on the
estimated coefficients.

\subsection{\texorpdfstring{\textbf{Final Model
Selection}}{Final Model Selection}}\label{final-model-selection}

TBD

\section{Results}\label{results}

\subsection{Kaplan-Meier Analysis}\label{kaplan-meier-analysis}

\begin{itemize}
\tightlist
\item
  \textbf{Plot}: Survival curves for the two bathing methods (routine
  vs.~antimicrobial).
\item
  \textbf{Log-rank test}: p-value for differences in survival times.
\end{itemize}

\subsection{Cox Proportional Hazards
Model}\label{cox-proportional-hazards-model-1}

\begin{itemize}
\tightlist
\item
  \textbf{Significant covariates}: Report hazard ratios, confidence
  intervals, and p-values.
\item
  \textbf{Interpretation}: Discuss the clinical meaning of these
  results.
\end{itemize}

\subsection{Time-Dependent Predictors}\label{time-dependent-predictors}

\begin{itemize}
\tightlist
\item
  \textbf{Surgical excision and antibiotics}: Analyze their time-varying
  effects on infection risk.
\end{itemize}

\section{Discussion}\label{discussion}

Summarize findings and their clinical implications. Discuss limitations
and future research directions.

\section{Conclusion}\label{conclusion}

Provide actionable recommendations based on the analysis.

\section{Appendices}\label{appendices}

Include additional plots, diagnostic checks, and R code if necessary.

\bibliographystyle{plain}
\bibliography{bibliography.bib}



\end{document}
